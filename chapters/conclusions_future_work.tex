\chapter{Conclusions and Future Work}\label{chp:conclusions}


\section{Concluding Remarks}\label{sec:conclusions}

Traditional peer-to-peer systems require all nodes in a conversation to have upload and download capacity of at least $(n-1) \times b$, where $n$ is the number of users in a conversation, and $b$ the minimal bitrate of a video stream. This requirement grows linearly with the number of users in a conversation, quickly becoming a limiting factor. Other solutions like Google Hangouts route all calls through the service provider's data centers, sacrificing latency for an upload requirement of only $b$. The approach presented in this thesis merges the best of these properties, allowing users in conversations with sufficient resources to route all their calls in a peer-to-peer to manner, while directing only constrained users through data centers, or through other peers with excess resources.

The approach in this thesis can also improve pure peer-to-peer conversations, by finding an optimal routing of video much faster than what browsers do through trial and error today. Providers adapting this work could instrument the browser directly, allow everyone to establish connections with bandwidth adjusted to fit them from the start without a long period to reach steady state.

Current browsers were shown to not handle constrained environments satisfactorily, either failing entirely or taking a very long time to establish decent connections to new nodes. The approach outlined could enable existing service providers to serve markets that are currently incapable of communicating over peer-to-peer services. Centralized providers could adapt the approach to push high-capacity conversations over peer-to-peer, to reduce the need for servers to manage the traffic.


\section{Future Work}\label{sec:future_work}

This thesis only presented a light sample implementation of the approach, which does not conclusively demonstrate that the approach is feasible. Some points below are listed that remains before this can be established.

\subsection{Implementation}

The system has not been tested in any actual video conversations and does not currently employ all the features described in \autoref{chp:suggested-solution} to achieve a fair routing. Implementing the missing features and testing the solution in actual conversations needs to be done.

As discussed in \autoref{sec:dynamic-conversations}, conversations are unlikely to be static, and will probably in many cases greatly change topology as the conversation progresses. This is inevitable for conversations with more than two participants, as the system will first solve for a $n=2$ system, and then later have to solve for a $n=3$ system when the next node joins. The system as proposed in this paper is completely oblivious to the state of a conversation, it simply takes in a configuration of nodes, and outputs an efficient routing of commodities. Adapting the system to support either change-sets to an existing topology, maybe prioritizing not interrupting existing connections when new nodes enter a conversation, or just recomputing from scratch, will require both the provider and the nodes to listen to notifications from others about topology changes. Incorporating this into the system in a transparent way for the user remains to be done in future work.


\subsection{Other Limiting Factors}

Bandwidth and latency were the only limiting factors we included in the sample implementation, but there's a ton of other variables that could influence desired topology, such as CPU, battery, screen size, link packet loss, link jitter, and user profiles ("Only do peer-to-peer, this is a confidential call").

None of these factors were considered for these first tests of a dynamic topology system, but should be considered for future work. Decisions could be based on data, but allow user overrides, such as prompting the user whether to optimize for performance, which will kill the battery in 10 minutes, or slightly degrade performance by routing video through a repeater/transcoder to reduce local CPU consumption to last another 20 minutes. Or just take the decision and silently notify the user. The degree of user autonomy in these cases are left to implementations to decide.


\subsection{Integrate Test Cases With Browsers}

If the entire test suite is made automatic, tests could easily be run by browser vendors to continuously assess how they perform in constrained environments. This also requires that Firefox exposes timing data from the RTCP stream through the \texttt{getStats} API, and preferably that the API gets standardized.

